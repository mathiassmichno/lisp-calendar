\chapter{Status}

This calendar Lisp language is both runnable and satisfies all parts of the given assignment in regard for functionalities and such.
One shortcoming in regard to the required functionalities is that input dates and times must be in the form of Unix timestamps, more on this in the \textbf{Overview} chapter.

\medskip
The program is written in the racket version of Scheme, without use of any libraries besides the scheme base library.
Racket's version of Scheme, resembles the \textit{R6RS} standard, with some minor deviations.
All development was done on \textit{macOS Sierra} in \textit{DrRacket} version 6.6.

\medskip
The program works as prescribed, however, some assumptions were made during development, which will be elaborated in the \textbf{Overview} chapter of this report.

For testing and presentation purposes, a secondary file named \texttt{test\_calendar} has been made, which contains an example of a calendar.
This test calendar file is loaded and evaluated when running the main file named \texttt{list\_calendar.rkt}.
Moreover, running the aforementioned program will output a html file with the same name as the test calendar.
