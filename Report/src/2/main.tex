\chapter{Overview}
Firstly some assumptions about the assignment will be presented, followed by a presentation of the calendar and appointment structures.
Lastly some shortcomings of the program will be described.

Moreover the source code in the \texttt{lisp\_calendar.rkt} file has been commented such that every function is described.

\section{Assumptions}
This calendar Lisp language makes some assumptions regarding the assignment:
\begin{itemize}
    \item A calendar does not require content, i.e.~appointment or calendars, to be valid.\\
        However, in the internal representation a calendar with no content simply contains an empty list.
    \item Functionality such as: \texttt{find-appointments}, \texttt{find-first-appointment}, and\\
        \texttt{find-last-appointment}, will find appointments in a calendar and any of said calendar's sub--calendars, and so forth.
    \item The time--format of appointments is defined as Unix timestamps, i.e. the number of seconds since January 1st 1970 at 00:00:00 GMT, not counting leap seconds\\
        This means that local time and date is not considered, and neither are dates before January 1st 1970.
    \item Furthermore while the presentation of the calendar will contain ``civil'' time and date, i.e. regular formatted dates and times,
        all appointments will be made using the raw Unix timestamps.
    \item Functionality which removes appointments or calendars from a calendar,\\
        i.e.~\texttt{remove-from-calendar}, \texttt{remove-appointments}, and \texttt{remove-calendars} must retain the structure of the input calendar.\\
        This means that the input calendar must not be flattened in any way shape or form.
    \item However, when adding or removing from a calendar, the order of the content is not guaranteed to be the same as in the source.
\end{itemize}

\section{Structure of Calendars and Appointments}
The internal representation of both a \textbf{calendar} and an \textbf{appointment} is based on a simple list structure.

\medskip
An appointment is defined as: \textit{A list containing exactly three elements in the following order; a string representing the title, a positive integer representing the starting time in Unix time, and another integer representing the ending time in Unix time. Moreover the starting time must be smaller than the ending time.}
If this is satisfied the list is a valid appointment, and will be matched be the \texttt{appointment?} predicate.

\medskip
A calendar is defined as: \textit{A list containing exactly two elements in the following order; a string representing the title, and a list representing the content of the calendar. Furthermore the list of content must only contain valid appointments or other valid calendars.}
A list satisfying this will be matched by the \texttt{calendar?} predicate as a valid calendar.

\medskip
Both appointments and calendars can be created using the \texttt{appointment} and \texttt{calendar} function respectively.
These function are guaranteed to evaluate to valid, well--formed structures.

\section{Shortcomings}
Because of to the choice of using Unix timestamps to represent date and time, a rather large portion of the program is an implementation of functions to transform Unix timestamps to human readable date--time strings.

Due to the time restriction of three days, I choose to focus on other functionalities as well as general polish, instead of implementing the reverse transformation.
That being said, a significant portion of the existing functions could be leveraged in such an implementation.

\medskip
Another weakness of the program is, the inability to present something else than an agenda in html.
Ideally a week-- or month--calendar would be implemented, and therefore some functions I thought would be useful
are present in the source code; such as \texttt{range-in-month}.



